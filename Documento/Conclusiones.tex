%!TEX root = main.tex
Para finalizar, daremos unas breves reflexiones del proyecto junto a posibles direcciones futuras.
\begin{itemize}
    \item Se pudo evidenciar como de manera teórica los códigos de Reed-Solomon, al tener una fundamentación más algebraica, en su enfoque como un cádigo cáclico, resulta mucho más complicado y lleno de errores en el proceso implementación de la decodificación que el caso de los códigos de Hamming.
    \item Una de las ventajas de usar Reed-Solomon, es que al aprovechar las propiedades algebraicas de los polinomios, podemos reducir la memoria usada, ya que almacenar matrices tiene mucho mas costo. Pero este método tiene sus desventajas y es que depende del mensaje editado y el algoritmo de decodificación usado, puede que palabras que teóricamente se puedan decodificar, la implementación no pueda hacerlo.
    \item Debido a aquellas cadenas que resultaba imposible decodificar, una posible dirección futura seria realizar una implementación del código más robusta  e intentar implementar algunos de los algoritmos de decodificación para poder ver el efecto que puede tener en las cadenas donde presenta error y teóricamente no debería hacerlo.
    \item Debido a lo recientes que son los resultados en este área de investigación, las ideas tratadas en este proyecto son solo un pequeño acercamiento hacia las analogias que se pueden hacer. Por lo que para continuar por esta línea se podría primero ahondar más en los metodos usados y buscar que método resultados más eficientes, también se podría estudiar estructuras más robustas de información, debido a que en el proyecto estudiamos cadenas simples, no uniones de estas. Este enfoque podria ser aún más enriquecedor.
\end{itemize}
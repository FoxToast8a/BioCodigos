%!TEX root = main.tex

La Teoria de la Codificacion y la de la informacion, vista desde los lentes de la matematica son ideas muy nuevas y se podria decir que dentro del esquema matematico se podrian categorizar como recientes o modernas. Si lo vemos desde estos ojos la implementacion de cadenas biologicas para el almacenamiento de informacion es aun mas novedoso, debido a que los primeros aportes fueron desarrollados a finales del siglo XX e inicios del siglo XXI.\\

Uno de los principales intereses de este campo relativamente emergente es estudiar los procesos biologicos y ver de que maneras se puede lograr el almacenamiento de datos en el mismo. Este tipo de enfoque es muy ambicioso y se requieren muchas herramientas mas avanzadas por lo que el proposito de este trabajo sera hacer un primer acercamiento para estudiar estos procesos, viendo como se comportan los procesos biologicos, el paralelo entre estos y los codigos correctores de errores y unas cuantas implementaciones para simular y visualizar los mismos.
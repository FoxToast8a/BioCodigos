%!TEX root = main.tex

\subsection{Reed-Solomon aplicado al ADN}
Dado el hecho que un mensaje puede ser visto por medio de $ASCII$, es natural empezar a preguntarnos por por un campo finito con $256$ elementos, es decir nuestro punto de partida sera $GF(D)$, con $D=2^8.$ Recordemos que este cuerpo se puede construir consiguiendo un polinomio irreducible de grado $8$ sobre $GF(2)$. Luego como cada elemento esta dado por el residuo, tenemos polinomios de grado $7$ o menos, así podemos escribir estos residuos simplemente como cadenas de los coeficientes.
$$GF(D)=\{a_0a_1\ldots a_7: a_i\in GF(2), 0\leq i\leq 7\}.$$